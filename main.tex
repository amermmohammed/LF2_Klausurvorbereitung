\documentclass[a4paper]{article}
\usepackage[top=2cm, bottom=2cm, left=1.5cm, right=1.5cm]{geometry}
\usepackage[german]{babel}
\usepackage[utf8]{inputenc}
\usepackage[T1]{fontenc}
\usepackage{eso-pic}
\usepackage{fontspec}   %%custom font
\usepackage{graphicx}   %%images
\setmainfont{[MahaloBrother.ttf]} %%costumfont
\usepackage{sectsty}    %% colored sections
\usepackage{datetime}
\usepackage{multicol}
\usepackage[usenames,dvipsnames]{color}
\usepackage{tikz}
\usetikzlibrary{matrix}
\usepackage{caption}
\usepackage{wrapfig}
\usepackage{float}
\usepackage[most]{tcolorbox}
%%%%
%%% add lines to page  %%%
\usepackage{tikzpagenodes}
\usepackage{lipsum}
\usepackage{background}
%%%%
\usepackage[dvipsnames]




\usepackage{hyperref}
\hypersetup{
    colorlinks=true,
    linkcolor=blue,
    filecolor=magenta,      
    urlcolor=cyan,
    pdftitle={Overleaf Example},
    pdfpagemode=FullScreen,
    }


\usepackage{fancyhdr}

\pagestyle{fancy}
\fancyhf{}
\lhead{ITB16-LF2}
\fancyhead[CE,CO]{LF2-Klausur}
\fancyfoot[CE,CO]{\Large{LF2-Klausur Vorbereitung}}
\rhead{\today}
\rfoot{Seite \thepage}
\lfoot{\href{mailto://ammohammed@schueler.berufskolleg.de}{Amer M. Mohammed}}


\usepackage{smartdiagram}
\usepackage{\dtklogos}

\usepackage{listings}
\usepackage[table,xcdraw]{xcolor}

\hypersetup{
    colorlinks=true,
    linkcolor=blue,
    filecolor=magenta,      
    urlcolor=cyan,
    pdftitle={Overleaf Example},
    pdfpagemode=FullScreen,
    }
    
\usepackage{amsmath}



\usepackage{xfakebold}

\newcommand{\fbseries}{\unskip\setBold\aftergroup\unsetBold\aftergroup\ignorespaces}
\makeatletter
\newcommand{\setBoldness}[1]{\def\fake@bold{#1}}
\makeatother
\setBoldness{0.4}%

\usepackage{microtype}
\usepackage{vwcol}
\chapterfont{\color{blue}}  % sets colour of chapters
\sectionfont{\color{cyan}}  % sets colour of sections
\subsectionfont{\color{violet}}
\subsubsectionfont{\color{purple}}


%% colors 

\definecolor{notepadrule}{RGB}{217,244,244}
\definecolor{codegreen}{rgb}{0,0.6,0}
\definecolor{codegray}{rgb}{0.5,0.5,0.5}
\definecolor{codepurple}{rgb}{0.58,0,0.82}
\definecolor{backcolour}{rgb}{0.95,0.95,0.92}


\tikzset{ 
    table/.style={
        matrix of nodes,
        row sep=-\pgflinewidth,
        column sep=-\pgflinewidth,
        nodes={
            rectangle,
            draw=blue,
            align=left
        },
        minimum height=1.5em,
        text depth=0.5ex,
        text height=2ex,
        nodes in empty cells,
%%
        every even row/.style={
            nodes={fill=purple!20}
        },
        column 1/.style={
            nodes={text width=2em,font=\bfseries}
        },
        row 1/.style={
            nodes={
                fill=cyan,
                text=white,
                font=\bfseries
            }
        }
    }
}


\lstdefinestyle{mystyle}{
    backgroundcolor=\color{backcolour},   
    commentstyle=\color{codegreen},
    keywordstyle=\color{magenta},
    numberstyle=\tiny\color{codegray},
    stringstyle=\color{codepurple},
    basicstyle=\ttfamily\footnotesize,
    breakatwhitespace=false,         
    breaklines=true,                 
    captionpos=b,                    
    keepspaces=true,                 
    numbers=left,                    
    numbersep=5pt,                  
    showspaces=false,                
    showstringspaces=false,
    showtabs=false,                  
    tabsize=2
}

\lstset{style=mystyle}





\begin{document}
    \begin{titlepage}
        \AddToShipoutPictureBG*{\includegraphics[width=\paperwidth,height=\paperheight]{media/bglf2}}

\begin{figure}[!htb]
    \centering
    \includegraphics[width=7cm]{logo-3}\label{fig:figure}
\end{figure}

\begin{center}
    \color{white}\Huge{\colorbox{BurntOrange}{\fbseries  LF2}}
    \vspace{5mm}
    \\ \color{black}\small{\colorbox{white}{\fbseries Berufskolleg Hilden des Kreises Mettmann}}
    \vspace{5mm}
    \\  \color{white}\Huge{\colorbox{BurntOrange}{Amer Malik Mohammed}}
\end{center}

\vspace{15mm}
\begin{center}
    {\color{white}\LARGE{\textbf{\vspace{30mm}} \colorbox{BurntOrange}{ITB16}\\\vspace{5mm}\colorbox{BurntOrange}{Supervisorin: Herr Epping}}}
\end{center}



\color{white}\centering\large\colorbox{BurntOrange}{letzte Bearbeitung}

\color{white}\centering\large\colorbox{BurntOrange}{\today\hspace{0.2cm}\currenttime}

    \end{titlepage}

%%% lines to page %%%%
    \backgroundsetup{
contents={%   
  \begin{tikzpicture}
    \foreach \fila in {0,...,44}
    {
      \draw [line width=1pt,color=notepadrule] 
      (current page.west|-0,-\fila*16pt) -- ++(\paperwidth,0);
    }
    \draw[overlay,red!70!black,line width=1pt]
      ([xshift=-1pt]current page text area.west|-current page.north) --  
      ([xshift=-1pt]current page text area.west|-current page.south);
  \end{tikzpicture}%
},
scale=1,
angle=0,
opacity=1
}
%%% end lines to page %%%


    \section{Kühlsysteme}\label{sec:kuehlsysteme}

    \subsection{\color{red}Nenne PC-Komponenten, die gekühlt werden müssen?}\label{subsec:color{red}nenne-pc-komponenten-die-gekühlt-werden-müssen?}

    \paragraph{\color{codegreen}Alle Komponenten, die Strom verbrauchen}
    \begin{itemize}
        \color{magenta}
        \item CPU
        \item Motherboard
        \item Grafikkarte
        \item Netzteil
        \item Arbeitsspeicher
        \item Festplatte
    \end{itemize}

    \subsection{\color{red}beschreibe die Funktionsweise von unterschiedlichen Kühlarten}\label{subsec:color{red}beschreibe-die-funktionsweise-von-unterschiedlichen-kühlarten}

    \paragraph{\color{codegreen}Luftstroms}
    \begin{itemize}
        \color{magenta}
        \item Kalte Luft ins Gehäuse
        \item Warme Luft raus
    \end{itemize}

    \paragraph{\color{codegreen}Wasserstrom}
    \begin{itemize}
        \color{magenta}
        \item  Kaltes Wasser zu der Hardware gepumpt
        \item Warmes Wasser wird am Radiator vom Lüfter gekühlt
    \end{itemize}

    \subsection{\color{red}nenne Vor- und Nachteile der unterschiedlichen Kühlarten}\label{subsec:color{red}nenne-vor--und-nachteile-der-unterschiedlichen-kühlarten}

    \subsubsection{\color{codegreen}Luftstroms}
    \begin{center}
        \begin{tabular}{cclll}
            \cline{1-2}
            \multicolumn{1}{|c|}{\textbf{Vorteile}}                          & \multicolumn{1}{c|}{\textbf{Nachteile}}                      &  &  &  \\ \cline{1-2}
            \multicolumn{1}{|c|}{{\color[HTML]{32CB00} Preis}}               & \multicolumn{1}{c|}{{\color[HTML]{F56B00} Weniger Leistung}} &  &  &  \\ \cline{1-2}
            \multicolumn{1}{|c|}{{\color[HTML]{32CB00} Leichter einzubauen}} & \multicolumn{1}{c|}{{\color[HTML]{F56B00} Staubfänger}}      &  &  &  \\ \cline{1-2}
            \multicolumn{1}{l}{}                                             & \multicolumn{1}{l}{}                                         & & &
        \end{tabular}
    \end{center}

    \subsubsection{\color{codegreen}Wasserkühlung}
    \begin{center}
        \begin{tabular}{|c|clll}
            \cline{1-2}
            \textbf{Vorteile}                                            & \multicolumn{1}{c|}{\textbf{Nachteile}}                          & & & \\ \cline{1-2}
            {\color[HTML]{32CB00} Mehr Leistung}                         & \multicolumn{1}{c|}{{\color[HTML]{F56B00} Komplizierter Einbau}} &  &  &  \\ \cline{1-2}
            {\color[HTML]{32CB00} Leiser}                                & \multicolumn{1}{c|}{{\color[HTML]{F56B00} Teurer}}               &  &  &  \\ \cline{1-2}
            \multicolumn{1}{|l|}{{\color[HTML]{32CB00} CPU Übertaktung}} & \multicolumn{1}{l}{}                                             &  &  &  \\ \cline{1-1}
        \end{tabular}
    \end{center}


    \section{Ergonomie}\label{sec:ergonomie}

    \subsection{\color{red}Nenne Größen der Temperatur, der Lautstärke und der Luftfeuchtigkeit für ein gutes Raumklima}\label{subsec:color{red}nenne-größen-der-temperatur-der-lautstärke-und-der-luftfeuchtigkeit-für-ein-gutes-raumklima}
    \begin{itemize}
        \color{magenta}
        \item Angenehme Raumtemperatur 21 bis 22° Celsius
        \item Im Sommer Obergrenze von 26° Celsius
        \item relative Luftfeuchtigkeit sollte 50 – 65 \% betragen
        \item Ideal sind Fenster zum Lüften und regelmäßige Stoßlüftung.
        \item Zum Schutz vor Sommerhitze sind Sonnenschutzvorrichtungen notwendig.
        \item \color[HTML]{F56B00} Lärm
        \begin{itemize}
            \item Schallpegel in einem Büro: höchstens 30 – 40 dB
            \item vorwiegend Geistigen Tätigkeiten: max.
            55 dB
            \item einfachen oder mechanischen Büroarbeiten: max.
            60 dB
        \end{itemize}
    \end{itemize}

    \subsection{\color{red}Beschreibe die Anforderungen an den Arbeitsplatz für eine gesunde Sitzhaltung}\label{subsec:color{red}beschreibe-die-anforderungen-an-den-arbeitsplatz-für-eine-gesunde-sitzhaltung}
    \begin{itemize}
        \color{magenta}
        \item Bildschirmoberkante nicht oberhalb
        der waagerechten Blicklinie
        \item Geeigneten Sehabstand zum
        Monitor schaffen.
        \item Tastatur ca.
        10--15 cm von
        der Tischkante entfernt parallel
        aufstellen.
        \item Maus nicht mit gestrecktem Arm bedienen, Mauspad ggf.
        mit Handballenauflage.
    \end{itemize}

    \subsection{\color{red}Beschreibe, wie der Arbeitsplatz im Raum (zur Fensterseite) angeordnet sein sollte}\label{subsec:color{red}beschreibe-wie-der-arbeitsplatz-im-raum-(zur-fensterseite)-angeordnet-sein-sollte}
    \begin{itemize}
        \color{magenta}
        \item Aufstellung des Tisches, so dass
        Blickrichtung parallel zum Fenster
        verläuft.
        \item Monitor gerade vor sich aufstellen,
        keine verdrehten Körperhaltungen
        \item • bei mehreren Fenstern parallel
        zur tageslichtintensitäten Fensterseite
        sitzen.
    \end{itemize}

    \subsection{\color{red}Erkläre, Vorteile einer ergonomischen Maus und Tastatur}\label{subsec:color{red}erkläre-vorteile-einer-ergonomischen-maus-und-tastatur}

    \paragraph{\color{codegreen} Eine \color{red} ergonomische Tastatur \color{codegreen} hat eine ergonomische Form und ist häufig noch zusätzlich verstellbar, damit Sie in einer natürlichen, ergonomischen Haltung am Computer arbeiten können. Beschwerden an Fingern, Handgelenk, Unterarm und Schulter können so verhindert oder behoben werden.}

    \paragraph{\color{codegreen} Eine \color{red}ergonomische Maus \color{codegreen} ist vertikal aufgebaut, wodurch Elle und Speichel parallel übereinander stehen und somit Sehnen, Muskeln und Nerven so wenig wie möglich belastet werden.}


    \section{Prozessor}\label{sec:prozessor}
    \begin{wrapfigure}{r}{.3\textwidth}
        \centering
        \includegraphics[scale=.2]{media/sb-nb}
        \captionsetup{labelformat=empty}
        \caption{Northbridge/Southbridge-Chipsatz\\\color{red}\small{Der Northbridge-Teil des Chipsatzes steuert die Hochgeschwindigkeitskanäle, während die Southbridge die Geräte mit niedrigerer Geschwindigkeit steuert.}}
        \label{fig:bridges}
    \end{wrapfigure}

    \subsection{Beschreibe die Aufgaben der North- und Southbridge eines Chipsatzes}\label{subsec:beschreibe-die-aufgaben-der-north--und-southbridge-eines-chipsatzes}

    \subsubsection{Northbridge}

    \paragraph{\color{codegreen}Der Hochgeschwindigkeitsteil einer gemeinsamen Chipsatzarchitektur in einem Computer. Die Northbridge ist der Controller, der die CPU über den Frontside-Bus (FSB) mit dem Speicher verbindet. Es verbindet auch Peripheriegeräte über Hochgeschwindigkeitskanäle wie PCI Express. Die Northbridge kann einen Anzeigecontroller enthalten, wodurch die Notwendigkeit einer separaten Grafikkarte entfällt.}

    \paragraph{\color{red}Northbridge Verbindet die CPU mit:}
    \begin{itemize}
        \color{magenta}
        \item RAM
        \item Eingebaute Grafik
        \item PCI-Express (PCIe)
    \end{itemize}

    \subsubsection{Southbridge}

    \paragraph{\color{codegreen}Der Southbridge-Controller verarbeitet die restlichen I/O, einschließlich PCI-Bus, parallele und Serial ATA-Laufwerke (IDE), USB, FireWire, serielle und parallele Anschlüsse und Audioanschlüsse. Frühere Chipsätze unterstützten den ISA-Bus in der Southbridge. Beginnend mit den 8xx-Chipsätzen von Intel wurden Northbridge und Southbridge in Memory Controller und I/O Controller geändert}

    \paragraph{\color{red}Southbridge Verbindet die CPU mit:}
    \begin{itemize}
        \color{magenta}
        \item SATA-Laufwerke
        \item USB-Bus
        \item Eingebautes Audio
    \end{itemize}

    \subsection{Beschreibe die Aufbau eines modernen Chipsatzes mit einem Platform Controller Hub bzw. Fusion Controller Hub (ohne Northbridge)}\label{subsec:beschreibe-die-aufbau-eines-modernen-chipsatzes-mit-einem-platform-controller-hub-bzw.-fusion-controller-hub-(ohne-northbridge)}

    \subsubsection{Platform Controller Hub (PCH)}
    \paragraph{\color{codegreen}Im Jahr 2008 wurde mit der Einführung des Chipsatzes der Intel 5-Serie die Northbridge/Southbridge-Architektur durch die Platform Controller Hub (PCH)-Architektur ersetzt. In dieser Architektur wird die Southbridge-Funktionalität vom PCH-Chip verwaltet, der über das DMI direkt mit der CPU verbunden ist.\newline
    Die meisten Northbridge-Funktionen wurden in die CPU integriert, während der PCH die restlichen Funktionen zusätzlich zu den traditionellen Rollen der Southbridge übernahm. In der PCH-Architektur sind die RAM- und PCIe-Datenpfade direkt mit der CPU verbunden. Beispiele für x86-Architekturen, bei denen die Northbridge in die CPU integriert ist, sind Intels Sandy Bridge und AMDs Fusion.\newline}

    \begin{center}
        \begin{figure}[H]
            \centering
            \includegraphics[height=5cm]{media/PCHVSN-S-bridge}
            \captionsetup{label-format=empty}
            \caption{north- south bridge vs PCH architecture}\label{fig:architecture}
        \end{figure}
    \end{center}

    \subsubsection{Fusion controller hubs (FCH)}

    \paragraph{\color{codegreen}{\fbseries Die Fusion Controller Hubs ähneln in ihrer Funktion dem Platform Controller Hub (PCH) von Intel. } Für AMD APU-Modelle von 2011 bis 2016. AMD vermarktet seine Chipsätze als Fusion Controller Hubs (FCH) und implementiert sie 2017 zusammen mit der Veröffentlichung der Zen-Architektur in seiner gesamten Produktpalette. Davor verwendeten nur APUs FCHs, während ihre anderen CPUs noch eine Northbridge und Southbridge verwendeten.}

    \subsection{Beschreibe die Begriffe ALU, FPU und FLOPS}\label{subsec:beschreibe-die-begriffe-alu-fpu-und-flops}

    \subsubsection{arithmetic logic unit (ALU)}

    \paragraph{\color{codegreen} Eine arithmetisch-logische Einheit ist ein elektronisches Rechenwerk, welches in Prozessoren zum Einsatz kommt. { \fbseries Die ALU berechnet arithmetische und logische Funktionen.}}

    \paragraph{\color{red}Mögliche Rechenoperationen}
    \begin{itemize}
        \color{magenta}
        \item Addition
        \item Subtraktion (Negativ-Addition)
        \item Vergleich
        \item Multiplikation
        \item Division
    \end{itemize}

    \paragraph{\color{red}Mögliche logische Verknüpfungen}
    \begin{itemize}
        \color{magenta}
        \item AND, OR, XOR
        \item Bitverschiebung
    \end{itemize}

    \subsubsection{Floating Point Unit (FPU)}

    \paragraph{\color{codegreen} Eine Gleitkommaeinheit (FPU) ist ein Teil eines Computersystems, das speziell dafür ausgelegt ist, Operationen mit Gleitkommazahlen auszuführen. Typische Operationen sind Addition, Subtraktion, Multiplikation, Division und Quadratwurzel. Einige FPUs können auch verschiedene transzendente Funktionen wie exponentielle oder trigonometrische Berechnungen ausführen, aber die Genauigkeit kann sehr gering sein, sodass einige Systeme es vorziehen, diese Funktionen in Software zu berechnen.}

    \subsubsection{Floating Point Operations Per Second (FLOPS)}

    \paragraph{\color{codegreen} Die Anzahl der Gleitkommaoperationen, die eine Recheneinheit (Prozessor oder gesamtes Rechnersystem) pro Sekunde ausführen kann. FLOPS werden als Maßeinheit benutzt, um die Rechenleistung von Systemen zu beschreiben..}

    \subsection{Erläutere die technische Angabe der Fertigungstechnik(z.B.7nm)}\label{subsec:erläutere-die-technische-angabe-der-fertigungstechnik(z.b.7nm)}

    \paragraph{\color{codegreen}7 Nanometer bezieht sich auf die Größe der beteiligten Transistoren. Je kleiner der Transistor ist, desto mehr passt auf ein Stück Silizium und desto leistungsfähiger und komplexer können die aus diesen Transistoren aufgebauten Komponenten sein.}

    \paragraph{\large{TO DO}}

    \subsection{Erkläre den Nutzen und den Aufbau von Cache-Speicher}\label{subsec:erkläre-den-nutzen-und-den-aufbau-von-cache-speicher}

    \paragraph{\color{codegreen} Der Cache-Speicher oder Cache Memory ist eine chipbasierte Computerkomponente, die das Abrufen von Daten aus dem Speicher des Computers effizienter macht. Er dient als temporärer Speicherbereich, aus dem der Prozessor des Computers Daten leicht abrufen kann. Dieser temporäre Speicherbereich, der als Cache bezeichnet wird, steht dem Prozessor leichter und schneller zur Verfügung als der Hauptarbeitsspeicher (Main Memory) des Computers, normalerweise eine Form von DRAM.}
    \subsubsection{Arten von Cache Memory}

    \paragraph{\color{red}Cache-Speicher ist schnell und teuer. Traditionell wird er in „Ebenen“ (Levels) kategorisiert, die seine Nähe und Zugänglichkeit zum Mikroprozessor beschreiben. Es gibt drei allgemeine Cache-Ebenen:}
    \begin{itemize}
        \color{magenta}
        \item {\fbseries Der L1-Cache } oder primäre Cache ist extrem schnell, aber relativ klein und wird normalerweise als CPU-Cache in den Prozessor-Chip eingebettet.
        \item {\fbseries Der L2-Cache } oder sekundäre Cache ist oft umfangreicher als der L1-Cache.
        Der L2-Cache kann in die CPU eingebettet sein, oder er kann sich auf einem separaten Chip oder Ko-Prozessor befinden und über einen alternativen Hochgeschwindigkeits-Systembus verfügen, der den Cache und die CPU verbindet.
        Auf diese Weise wird er nicht durch den Verkehr auf dem Hauptsystembus verlangsamt.
        \item {\fbseries Der Cache } der Ebene 3 (L3) ist ein spezialisierter Arbeitsspeicher, der entwickelt wurde, um die Leistung von L1 und L2 zu verbessern.
        L1 oder L2 können wesentlich schneller sein als L3, obwohl L3 normalerweise doppelt so schnell wie DRAM ist.
        Bei Mehrkernprozessoren kann jeder Kern (Core) über einen dedizierten L1- und L2-Cache verfügen, aber sie können sich einen L3-Cache teilen.
        Wenn ein L3-Cache auf eine Anweisung verweist, wird er normalerweise auf eine höhere Cache-Ebene angehoben.
    \end{itemize}

    \subsection{Beschreibe die Programmiersprache Assembler}\label{subsec:beschreibe-die-programmiersprache-assembler}
    \paragraph{\color{codegreen} Assemblersprache ist jede Low-Level-Programmiersprache, in der eine sehr starke Entsprechung zwischen den Anweisungen in der Sprache und den Maschinencodeanweisungen der Architektur besteht.
    Assemblercode wird durch ein Dienstprogramm, das als Assembler bezeichnet wird, in ausführbaren Maschinencode umgewandelt. }


    \section{Monitore}\label{sec:monitore}

    \subsection{Unterscheide die Panelarten TN, VA, IPS anhand von technischen Eigenschaften voneinander}\label{subsec:unterscheide-die-panelarten-tn-va-ips-anhand-von-technischen-eigenschaften-voneinander}
    \paragraph{\color{codegreen} Bei Flüssigkristall Monitor strahlt Licht durch rote, grüne und blaue Farbfilter. Die Flüssigkristalle lassen mal mehr, mal weniger dieses Licht durch und mischen so die Farben zusammen.\\
    Bei \color{red}{\fbseries TN-Panels }\color{codegreen} wechseln die Kristalle die Ausrichtung von horizontal auf vertikal. Bei \color{red}{\fbseries AV-Panels }\color{codegreen} genau umgekerht. Bei \color{red}{\fbseries IPS-Panels }\color{codegreen} drehen sich die Kristalle auf gleicher Ebene um 90 Grad.}
    \begin{center}
        \begin{figure}[H]
            \centering
            \includegraphics[height=5cm]{media/panals}
            \captionsetup{labelformat=empty}
            \caption{TN vs VA vs IPS}
            \label{fig:panals}
        \end{figure}
    \end{center}

    \subsection{Erkläre, wie ein Pixel aufgebaut ist?}\label{subsec:erkläre-wie-ein-pixel-aufgebaut-ist?}

    \paragraph{\color{codegreen} Ein Pixel ist aus mehreren Subpixeln aufgebaut, die auf Monitoren und anderen Displays in Rot, Grün und Blau (RGB) wiedergegeben werden. Je nach Zusammensetzung dieser drei Farben entsteht ein bestimmter Farbton, der in Kombination mit meist Millionen anderer Bildpunkte das Gesamtbild ergibt.}

    \subsection{Beschreibe die LCD- und OLED-Technik mit einfachen Worten}\label{subsec:beschreibe-die-lcd--und-oled-technik-mit-einfachen-worten}
    \begin{multicols}{2}
        \subsubsection{LCD-Bildschirm}
        \begin{itemize}
            \color{magenta}
            \item Beim LCD-Bildschirm werden Flüssigkristalle eingesetzt.
            Jeder dieser Kristalle stellt einen Bildpunkt, also ein Pixel, dar.
            \item Hinter den Flüssigkristallen befindet sich die Hintergrundbeleuchtung.
            Entweder durch LEDs, die aus den Ecken heraus leuchten oder durch Leuchtstoffröhren direkt hinter den Kristallen.
            \item Die Kristalle können einzeln ausgerichtet werden, so dass sie weniger oder mehr Licht durchlassen und die jeweilige Farbe wiedergeben.
        \end{itemize}

        \subsubsection{OLED-Bildschirm}
        \begin{itemize}
            \color{magenta}
            \item Der OLED-Bildschirm benötigt keine Hintergrundbeleuchtung.
            Stattdessen leuchtet jedes OLED für sich, jeder Bildpunkt ist also eine Lichtquelle.
            \item Das funktioniert mit zwei Elektroden, von denen eine transparent ist.
            Zwischen den beiden Elektroden befinden sich verschiedene organische Halbleiterschichten.
            \item Wird Strom durch die Elektroden geschickt, leuchten die Halbleiterschichten.
            Die Stromstärke reguliert die Helligkeit.
        \end{itemize}
    \end{multicols}
    \subsection{Erkläre, technische Angaben zu einem Monitor (Bildwiederholungsrate, Auflösung, Pixeldichte,
        Größe (Zoll), Seitenverhältnis)}\label{subsec:erkläre-technische-angaben-zu-einem-monitor-(bildwiederholungsrate-auflösung-pixeldichte
    --------größe-(zoll)-seitenverhältnis)}

    \subsubsection{Bildwiederholungsrate (Bildwiederholfrequenz)}

    \paragraph{\color{codegreen} Die Bildwiederholfrequenz besagt, wie oft sich ein Vorgang pro Sekunde wiederholt. Ein 50-Hertz-Fernseher zeigt Bilder fünfzig Mal pro Sekunde, ein 100-Hertz-Gerät hundert Mal und so weiter.}

    \subsubsection{Auflösung}

    \paragraph{\color{codegreen} Die Auflösung eines Bildes wird in der Regel in „ppi“ (pixels per inch) angegeben und beschreibt, wie viele Pixel (digitale Bildpunkte) auf der Länge von einem inch/Zoll (2.54 cm) vorhanden sind.}

    \subsubsection{Pixeldichte}

    \paragraph{\color{codegreen} Die Pixeldichte von Displays ist ein Maß für den Grad des Auflösungsvermögens, wobei der Wert üblicherweise in dpi ausgedrückt wird. Dpi steht für „Punkte pro Zoll“ („Dots per Inch“, nicht pro Quadratzoll). Ein Zoll entspricht 2,54 Zentimeter.}

    \subsubsection{Größe (Zoll)}

    \paragraph{\color{codegreen} Ein Zoll ist etwas länger, als zweieinhalb Zentimeter, genau gilt: 1" = 2,54 cm. Das internationale Zoll wird als übliches Längenmaß vor allem noch in den USA verwendet sowie für festgelegte Größenangaben in der Technik.}

    \subsubsection{Seitenverhältnis}

    \paragraph{\color{codegreen} Unter Seitenverhältnis im weiteren Sinne versteht man das Verhältnis von mindestens zwei unterschiedlich langen Seiten eines Polygons. Meistens wird damit das Verhältnis der Breite eines Rechtecks zu seiner Höhe angegeben. Ein Quadrat hat das Seitenverhältnis 1:1.}
    \begin{center}
        \begin{figure}[H]
            \centering
            \includegraphics[height=5cm]{media/bs}
            \captionsetup{labelformat=empty}
            \caption{\color{red} Seitenverhältnis: das Verhältnis zwischen der Breite und der Höhe eines Bildes.}
            \label{fig:Seitenverhaeltnis}
        \end{figure}
        \begin{tabular}{|c|c|c|}
            \hline
            \textbf{Seitenverhältnis}   & \textbf{perfekt für}                                                   & \textbf{Welche Fernseher?}                                        \\ \hline
            {\color[HTML]{009901} 4:3}  & {\color[HTML]{CE6301} Alte Filme / Serien}                             & {\color[HTML]{3531FF} Alte Fernseher (häufig noch Röhrengeräte)}  \\ \hline
            {\color[HTML]{009901} 16:9} & {\color[HTML]{CE6301} Aktuelles Fernsehprogramm / Filme für Fernsehen} & {\color[HTML]{3531FF} Fast jeder moderne Flachbildfernseher}      \\ \hline
            {\color[HTML]{009901} 21:9} & {\color[HTML]{CE6301} Kinofilme}                                       & {\color[HTML]{3531FF} Wenige - meist Premium-Flachbildfernseher} \\ \hline
        \end{tabular}
    \end{center}


    \section{Softwarelizenzen}\label{sec:softwarelizenzen}

    \subsection{Was sind Software-Lizenzen?}\label{subsec:was-sind-software-lizenzen?}

    \paragraph{Es handelt sich um eine rechtsverbindliche Vereinbarung zwischen Endnutzer und Softwarehersteller. Durch die Lizenz werden die Nutzungsbedingungen bis ins Detail geregelt. Die Lizenz ist also ein Vertrag, mit dem Urheber die Rechte an ihrem geistigen Eigentum auf andere überträgt. Dies geht immer mit einer direkten oder indirekten Gegenleistung einher, oder die Übertragung findet nur unter bestimmten Bedingungen statt}

    \subsection{beschreibe, Lizenzmodelle und -arten mit wenigen Worten}\label{subsec:beschreibe-lizenzmodelle-und--arten-mit-wenigen-worten}
    \begin{center}
        \begin{tabular}{|l|l|}
            \hline
            \multicolumn{1}{|c|}{{\color[HTML]{00D2CB} Name der Lizenz}} & \multicolumn{1}{c|}{{\color[HTML]{00D2CB} Bedeutung}}                                         \\ \hline
            {\color[HTML]{CB0000} Freeware}                              & {\color[HTML]{3166FF} Kostenlose Nutzung, offene Sourcen}                                     \\ \hline
            {\color[HTML]{CB0000} Open Source}                           & {\color[HTML]{3166FF} Quellcode frei zugänglich, nicht immer kostenlos}                       \\ \hline
            {\color[HTML]{CB0000} Shareware}                             & {\color[HTML]{3166FF} Kostenlose Testung \& Verbreitung, meist beschränkte Version}           \\ \hline
            {\color[HTML]{CB0000} Donationware}                          & {\color[HTML]{3166FF} Spenden für Weiterentwicklung/ -betreibung}                             \\ \hline
            {\color[HTML]{CB0000} Standard Lizenz}                       & {\color[HTML]{3166FF} Entweder Gerät oder Account gebunden}                                   \\ \hline
            {\color[HTML]{CB0000} Abonnement basierte Lizenz}            & {\color[HTML]{3166FF} Kostenpflichtiges Abo, aber zeitlich beschränkt}                        \\ \hline
            {\color[HTML]{CB0000} EULA}                                  & {\color[HTML]{3166FF} Endbenutzerlizenzvertrag, festgelegte Nutzungsbedingungen}              \\ \hline
            {\color[HTML]{CB0000} Arbeitsstation Lizenz}                 & {\color[HTML]{3166FF} Nur für einen Computer, ein Back-Up meist erlaubt}                      \\ \hline
            {\color[HTML]{CB0000} Cloud basierte Lizenz}                 & {\color[HTML]{3166FF} Über Cloud jederzeit und überall zugreifbar}                            \\ \hline
            {\color[HTML]{CB0000} Aktivierungslizenz}                    & {\color[HTML]{3166FF} Lizenz zur Produktaktivierung}                                          \\ \hline
            {\color[HTML]{CB0000} Public Domain}                         & {\color[HTML]{3166FF} Kompletter Verzicht auf Urheberrechte, der Quellcode ist öffentlich}    \\ \hline
            {\color[HTML]{CB0000} Cardware}                              & {\color[HTML]{3166FF} Entwickler wünscht sich Postkarte von den Nutzern}                      \\ \hline
            {\color[HTML]{CB0000} Adware}                                & {\color[HTML]{3166FF} Software ist kostenlos, finanziert sich aber durch Werbung}             \\ \hline
            {\color[HTML]{CB0000} Kommerzielle Software-Lizenz}          & {\color[HTML]{3166FF} Nutzer erwirbt Nutzungsrechte an der Software, meist entgeltlich, kann} \\ \hline
            {\color[HTML]{CB0000} By Name}                               & {\color[HTML]{3166FF} auch gratis sein}                                                       \\ \hline
            {\color[HTML]{CB0000} No commercial use}                     & {\color[HTML]{3166FF} Kommerzielle Verwendung verboten}                                       \\ \hline
        \end{tabular}
    \end{center}

    \section{Green-IT}\label{sec:green-it}
    \subsection{Ziele der Green-IT}\label{subsec:ziele-der-green-it}
        \begin{itemize}
            \color{magenta}
            \item Reduzierung des Energieverbrauchs
            \item Recyclings und Wiederverwendung
            von Geräten
            \item Nutzung erneuerbarer Energien
            \item Nachhaltigkeit von Unternehmen zu
            verbessern
            \item Langlebige Produkte herstellen
        \end{itemize}
    \subsection{Erleuterung und Bennenug der Maßnahmen zur Reduzierung der Umweltbelastung}\label{subsec:erleuterung-und-bennenug-der-maßnahmen-zur-reduzierung-der-umweltbelastung}
        \begin{multicols}{2}
            \begin{enumerate}
                \color{magenta}
                \item Cloud-Hosting
                \begin{itemize}
                    \color{blue}
                    \item Reduziert den CO2-Ausstoß
                    \item Kostensenkung
                    \item weniger Geräte => weniger Energie verbraucht
                    \item Kunden verbrauchen 77\% weniger Server, 84\% weniger
                    Strom und reduzieren die Kohlenstoffemissionen um 88\%.
                \end{itemize}
                \color{magenta}
                \item Virtualisierung
                \begin{itemize}
                    \color{blue}
                    \item Senkung der Wartungskosten
                    \item Erhöhung der Sicherheit
                    \item Einfache Implementierung
                    \item Senkung der Energiekosten
                    \item Zentralisierte Verwaltung
                    \item Weniger Ausfallzeiten/höhere Produktivität
                \end{itemize}
                \color{magenta}
                \item REFURBISHING/RECYCLING
                \begin{itemize}
                    \color{blue}
                    \item Vermeidung von toxischer Verschmutzung
                    \item Vermeidet Elektroschrott
                \end{itemize}
                \color{magenta}
                \item Umweltschonende Hardware
                \begin{itemize}
                    \color{blue}
                    \item Kauf nur von nachhaltiger Hardware
                    \item Umweltfreundliche Labels
                    \item Verwendung von Hardware, die langlebiger ist
                \end{itemize}
                \color{magenta}
                \item Standby-Modus \& Geräte abschalten
                \begin{itemize}
                    \color{blue}
                    \item Konfiguration des Standby-Modus in allen Geräten
                    \item Unbenutzte Geräte ausschalten
                \end{itemize}
                \color{magenta}
                \item Nachhaltige Büros
                \begin{itemize}
                    \color{blue}
                    \item IT-Ausstattung dem individuellen Bedarf
                    anpassen
                    \item Papierloses Büro
                    \item Energiesparende Geräte kaufen
                    \item Mobile Arbeitsprozesse
                \end{itemize}
            \end{enumerate}
        \end{multicols}
    \subsection{Umwelt-Prüfzeichen mit den grundlegenden Zielen}\label{subsec:umwelt-prüfzeichen-mit-den-grundlegenden-zielen}
        \begin{multicols}{2}
            \begin{figure}[H]
                \centering
                \includegraphics[height=2cm]{media/estar}
                \captionsetup{labelformat=empty}
                \caption{Das Programm wurde 1992 von der US-
                Umweltschutzbehörde ins Leben gerufen,
                    wobei Computer und Monitore diese
                    Auszeichnung erhielten. Heutzutage findet
                    man das Zeichen auch auf Großgeräten,
                    Beleuchtungseinrichtungen und anderen
                    elektronischen Geräten.}
                \label{fig:estar}
            \end{figure}
            \begin{figure}[H]
                \centering
                \includegraphics[height=2cm]{media/tco-certified}
                \captionsetup{labelformat=empty}
                \caption{Zeichnet Produkte wie Monitore,
                    Drucker oder Mobiltelefone aus, die
                    benutzer- und umweltfreundlich und
                    energieeffizient sind}
                \label{fig:tco}
            \end{figure}
            \begin{figure}[H]
                \centering
                \includegraphics[height=2cm]{media/ECOLABEL}
                \captionsetup{labelformat=empty}
                \caption{Das Zeichen wird für Produkte und
                Dienstleistungen vergeben, die eine
                geringere Umweltbelastung aufweisen als
                vergleichbare Produkte. Das EU-
                Umweltzeichen soll den Verbrauchern die
                Möglichkeit geben, umweltfreundlichere und
                gesündere Produkte zu erkennen.}
                \label{fig:ECOLABEL}
            \end{figure}
            \begin{figure}[H]
                \centering
                \includegraphics[height=2cm]{media/TUEV-Rheinland-Logo2.svg}
                \captionsetup{labelformat=empty}
                \caption{Das Zertifikat "Energieeffizientes
                Rechenzentrum" garantiert, dass sich das
                Unternehmen der Nachhaltigkeit verpflichtet
                fühlt.}
                \label{fig:TUEV}
            \end{figure}
        \end{multicols}
\end{document}